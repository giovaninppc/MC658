\documentclass[conference]{IEEEtran}
\usepackage{cite}
\usepackage[portuges,brazil]{babel}
\usepackage{amsmath,amssymb,amsfonts}
\usepackage{siunitx}
\usepackage{algorithmic}
\usepackage{graphicx}
\usepackage{textcomp}
\usepackage{hyperref}
\usepackage{listings}
\usepackage[toc,page]{appendix}
\usepackage[utf8]{inputenc}

\def\BibTeX{{\rm B\kern-.05em{\sc i\kern-.025em b}\kern-.08em
    T\kern-.1667em\lower.7ex\hbox{E}\kern-.125emX}}

\begin{document}

\title{Lab 2: Problema dee Rotas com Pesos Cumulativos\\
\large .\\
\large MC658 - Projeto e Análise de Algoritmos III 2S2019\\
\large Professor: Flávio Keidi Miyazawa}

\newcommand{\email}[1]{\href{mailto:#1}{#1}}

\author{
    \IEEEauthorblockN{Giovani Nascimento Pereira}
    \IEEEauthorblockA{
    \email{giovani.x.pereira@gmail.com} \\
    168609
    }
}

\maketitle

\section{Definição do problema}

    Considere um caminhão sai de uma cidade de origem $s$ até uma cidade de destino $t$, transportando carga ao longo do caminho.
    Cada cidade possui um produto $i$ que pode ser coletado para transporte, e o produto tem um peso $w_i$ e lucro $l_i$ de ser entregue.
    Quando um item é coletado, ele aumenta em $w_i$ o peso do caminhão. Cada arco $(i,j)$ tem um valor $c_a$ que indica o custo de transporte de cada unidade de peso naquele arco.

    Seja,

    \begin{itemize}
        \item $P$ peso do caminhão
        \item $C$ capacidade de carga adicional do caminhão
        \item Conjunto de cidades $V$
        \item Conjunto de estradas $A$ onde $(i,j) \in A$ indica que existe estrada entre as cidades $i$ e $j$
        \item Cidade de origem $s$
        \item Cidade de destino $t$
        \item $w_i$ o peso do produto $i$
        \item $l_i$ o lucro do produto $i$
        \item $c_a$ custo de transporte de $1$ unidade de peso no arco $a$
    \end{itemize}

    O problema consiste eem encontrar qual a rota que trará o maior lucro ao final. Ou seja, maximizar a relação de lucro obtido com o transporte dos itens menos o gasto com o transporte.

    Se tomarmos um arco $a$, onde o caminhão carrega itens $S_a \in V$, o custo de transporta em $a$ será:

    \begin{equation}
        \text{Custo de transporte em } a = (P + \Sigma_{i \in S_a} w_i) * c_a
    \end{equation}

    Ou seja, o peso total do caminhão (peso base $P$ mais peso dos itens sendo transportados) multiplicado pelo custo $c_a$.

    Então podemos resumir, que dado um caminho $\Rho$ de $s$ a $t$:

    \begin{equation}
        valor(\Rho) = \Sigma_{v \in V(\Rho)} - \Sigma_{(u,v) \in A(\Rho)} (P + w(\Rho, u))c_{(u,v)}
    \end{equation}

\section {Resolução}


\newpage
\section {Formulação}



\end{document}
