\documentclass[conference]{IEEEtran}
\usepackage{cite}
\usepackage[portuges,brazil]{babel}
\usepackage{amsmath,amssymb,amsfonts}
\usepackage{siunitx}
\usepackage{algorithmic}
\usepackage{graphicx}
\usepackage{textcomp}
\usepackage{hyperref}
\usepackage{listings}
\usepackage[toc,page]{appendix}
\usepackage[utf8]{inputenc}

\def\BibTeX{{\rm B\kern-.05em{\sc i\kern-.025em b}\kern-.08em
    T\kern-.1667em\lower.7ex\hbox{E}\kern-.125emX}}

\begin{document}

\title{Lab 2: Problema dee Rotas com Pesos Cumulativos\\
\large .\\
\large MC658 - Projeto e Análise de Algoritmos III 2S2019\\
\large Professor: Flávio Keidi Miyazawa}

\newcommand{\email}[1]{\href{mailto:#1}{#1}}

\author{
    \IEEEauthorblockN{Giovani Nascimento Pereira}
    \IEEEauthorblockA{
    \email{giovani.x.pereira@gmail.com} \\
    168609
    }
}

\maketitle

\section{Definição do problema}

    Considere um caminhão sai de uma cidade de origem $s$ até uma cidade de destino $t$, transportando carga ao longo do caminho.
    Cada cidade possui um produto $i$ que pode ser coletado para transporte, e o produto tem um peso $w_i$ e lucro $l_i$ de ser entregue.
    Quando um item é coletado, ele aumenta em $w_i$ o peso do caminhão. Cada arco $(i,j)$ tem um valor $c_a$ que indica o custo de transporte de cada unidade de peso naquele arco.

    Seja,

    \begin{itemize}
        \item $P$ peso do caminhão
        \item $C$ capacidade de carga adicional do caminhão
        \item Conjunto de cidades $V$
        \item Conjunto de estradas $A$ onde $(i,j) \in A$ indica que existe estrada entre as cidades $i$ e $j$
        \item Cidade de origem $s$
        \item Cidade de destino $t$
        \item $w_i$ o peso do produto $i$
        \item $l_i$ o lucro do produto $i$
        \item $c_a$ custo de transporte de $1$ unidade de peso no arco $a$
    \end{itemize}

    O problema consiste em encontrar qual a rota que trará o maior lucro ao final. Ou seja, maximizar a relação de lucro obtido com o transporte dos itens menos o gasto com o transporte.

\section {Resolução}

    A maior dificuldade desse problema, é a questão de ordem envolvida para o custo:

    Se fizermos um caminho em vértices $a \rightarrow b \rightarrow c$, onde cada vértice tem um item com valor e peso diferente, e cada aresta também tem um custo de transporte diferente, o valor adquirido ao passar por esse caminho seria o mesmo em qualquer permutação de $a$, $b$, $c$, mas não o custo de transporte.

    Então precisamos de alguma forma indicar que ao chegar no vértice $i$ exista, implicitamente, o peso do caminhão com seus itens ao chegar.

    Vamos tomar uma variável $F_{i,j} \in Z$ que acumula o peso do caminhão até o vértice $i$ (note que $F$ não é binária), e uma variável $Y_{i,j} \in \{0,1\}$ que indica que uma aresta $(i,j)$ pertence à solução, e uma variável $X_i \in \{0,1\}$ que indica que um vértice $i$ pertence à solução.

    Podemos descrever a relação que define $F$ como sendo:

    \begin{equation}
        \Sigma_{j \in V}F_{i,j} = \Sigma_{j \in V} P_{j,i} + X_i w_i \text{ , } \forall i \in V \\ {s,t}
    \end{equation}

    Ou seja, para um vértice $i$ qualquer, o custo das artestas que saem dele, é igual ao custo das arestas que entram nele somando $Y_i w_i$, que é o custo de transporte naquela aresta, mas apenas se ela pertence à solução.

\\

\section {Formulação}

    Sejam $F_{i,j} \in Z$ variável que indica o peso total do caminhão na aresta $(i,j)$, $Y_{i,j} \in \{0, 1\}$ variável indicadora que diz que a aresta $(i,j)$ pertence à solução e $X_i \in \{0, 1\}$ variável indicadora que diz que o vértice $i$ pertence à solução.

    Podemos definir a formulação do problema como otimizar a função:

    \begin{equation}
        \Sigma_{i \in V} (\Sigma_{j \in V} Y_{i,j}) l_i - \Sigma_{(i,j) \in A} F_{i,j}c_{i,j} + P c_{i,j} Y_{i,j}
    \end{equation}

    Onde $(\Sigma_{j \in V} Y_{i,j}) l_i$ mostra que se uma aresta incide em $i$, então o lucro do item $i$ pertence à solução, menos o custo de transporte em todas as arestas que também pertencem à solução.

    Sujeita a:

    \begin{equation}
        \Sigma_{i \in V} Y_{i,a} \leq 1, \forall a \in V - \{s,t\}
        \label{1}
    \end{equation}

    \begin{equation}
        \Sigma_{i \in V} Y_{a,i} \leq 1, \forall a \in V - \{s,t\}
        \label{2}
    \end{equation}

    \begin{equation}
        \Sigma_{i \in V} Y_{i,a} = \Sigma_{i \in V} Y_{a,i}, \forall a \in V - \{s,t\}
        \label{no-cicle}
    \end{equation}

    \begin{equation}
        \Sigma_{i \in V} Y_{s,i} = 1 \text{  e,  }  \Sigma_{i \in V} Y_{i,s} = 0
        \label{s}
    \end{equation}

    \begin{equation}
        \Sigma_{i \in V} Y_{t,i} = 0 \text{  e,  } \Sigma_{i \in V} Y_{i,t} = 1
        \label{t}
    \end{equation}

    \begin{equation}
        P + \Sigma_{i \in  V} (\Sigma_{j \in V} Y_{j,i}) w_i \leq C
        \label{capacity}
    \end{equation}

    \begin{equation}
        \Sigma_{j \in V}F_{i,j} = \Sigma_{j \in V} P_{j,i} + X_i w_i \text{ , } \forall i \in V \\ {s,t}
        \label{belong}
    \end{equation}

    \begin{equation}
        X_s = 0
        \label{ps}
    \end{equation}

    \begin{equation}
        X_j = \Sigma_{Y_{i,j}} \text{ , } \forall j \in V \\ \{\}
        \label{vertez}
    \end{equation}

    Onde,

    \begin{itemize}
        \item As restrições \ref{1} e \ref{2} impedem a existência de ciclos no caminho $\rho$ da solução, toda aresta tem no máximo um vértice de entrada e um de saída;

        \item A restrição \ref{no-cicle} garante que se entra uma aresta em um vértice que não seja $s$ ou $t$, também sai uma aresta desse vértice.

        \item \ref{s} garante que do nó inicial $s$ saia  exatamente 1 aresta e que nenhuma aresta atinja  $s$.
        \item \ref{t} garante que do nó final $t$ não saia nenhuma aresta e que exatamente 1 aresta incida em $t$.

        \item \ref{capacity} impede que se aloque mais itens que a capacidade do caminhão permite.

        \item e a restrição \ref{belong} é a formulação que garante que nosso $F$ trabalhe como um acumulador de pesos das arestas, e guarda o valor do peso do caminhão de $s$ até o vértice $j$.

        \item \ref{ps} faz com que o peso do caminhão comece zerado no primeiro vértice.

        \item e por fim \ref{vertez} faz com que um vértice pertença a solução, se existe uma aresta que incide nele.

    Note que, como \ref{s} e \ref{t} garantem o começo e fim do caminho, junto com a restrição \ref{no-cicle}, elas garantem a conectividade do caminho de $s$ a $t$.

    \end{itemize}


\end{document}
